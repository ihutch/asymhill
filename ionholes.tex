%\documentclass[pre]{revtex4-2}
%\documentclass[pre,twocolumn]{revtex4-2}
\documentclass[12pt]{article}  % Vanilla Latex.
%%%%%%%%%%%%%%%%%%%%%%%%%%%%%%%%%%%%%%%%%%%%%%%%%%%%%%%%%%%%%%%%%%%%%%
\usepackage{amssymb}
\usepackage{amsmath}
\usepackage{graphicx}
\usepackage{hyperref}
\usepackage{bm}
%%%%%%%%%%%%%%%%%%%%%%%%%%%%%%%%%%%%%%%%%%%%%%%%%%%%%%%%%%%%%%%%%%%%%%%%
\ifx\affiliation\undefined %Then we are not using a journal class
\def\affiliation#1{\date{\normalsize #1}}
\usepackage[margin=1in]{geometry}
\usepackage[sort&compress,numbers]{natbib}
\bibliographystyle{unsrtnat}
\fi
%%%%%%%%%%%%%%%%%%%%%%%%%%%%%%%%%%%%%%%%%%%%%%%%%%%%%%%%%%%%%%%%%%%%%%%%
\def\energy{{\cal E}}
\def\etothe#1{{\rm e}^{#1}}
\def\sech{{\,\rm sech}}
%%%%%%%%%%%%%%%%%%%%%%%%%%%%%%%%%%%%%%%%%%%%%%%%%%%%%%%%%%%%%%%%%%%%%%%% 

\begin{document}

\title{Ion Holes}
\author{I H Hutchinson}
\affiliation{Plasma Science and Fusion Center,\\ Massachusetts Institute of
  Technology,\\ Cambridge, Massachusetts, 02139, USA}

\ifx\altaffiliation\undefined\maketitle\fi % Latex documentclass.
\begin{abstract}
\end{abstract}
\ifx\altaffiliation\undefined\else\maketitle\fi  % Revtex

\section{Symmetric Maxwellian Background Equilibria}

Electron holes can, and often do, have speeds much higher than ion
thermal speed, allowing them to be considered unaffected by ion
interactions. When this is not the case they are considered ``slow''.
By contrast, ion holes exist only when the spread of the ion distribution
function $f_i(v)$ in the hole rest frame makes $f_i(0)\not=0$. The ion
velocity spread is much smaller for physical mass ratios than the
electron velocity spread; so ions are thus practically always slower than
electrons and ion holes are always ``slow'' in the sense that they reflect
some of the repelled species: electrons. The simplest way then to avoid net
reflection force on the hole is to suppose that the repelled species
velocity distribution is symmetric about the hole velocity, and we
begin by considering $f_e(v)$ to be an unshifted Maxwellian in the
frame of the hole. The electron density is then
$n_e=\etothe{\phi/\theta}$, where $\theta=T_e/T_i$.

Suppose the (attracted) ion velocity distribution consists of a
reference distribution that is Maxwellian for passing particles and is
flat (independent of $v$) for trapped particles, matching the
untrapped Maxwellian at their join where total energy in the hole
frame, $\energy$, is zero. This ``flat-trapped'' distribution $f_f(v)$
governs the attracted species when there is no ``hole'' in the
distribution. A self-sustaining hole generally depends upon there
being a trapped deficit relative to the flat-trapped
distribution. That is represented by an additional function
$\tilde f_i(v)$ which is negative for trapped particles, $\energy<0$,
(and zero for passing $\energy>0$). Then supposing the ion Maxwellian
distribution to have velocity shift $v_s$ in the hole frame,
\begin{equation}
  \label{eq:flattrap}
  f_f(v)={1\over \sqrt{2\pi}}\etothe{-(v-v_s)^2/2}
  ={1\over \sqrt{2\pi}}\exp\{-(\sqrt{{\rm max}(2[\energy+|\phi|],0)}-v_s)^2/2\}
\end{equation}
where $\energy=v^2/2+\phi(z)$, and $f_i(v)=f_f(v)+\tilde f_i(v)$ with all
velocities are expressed in the hole rest frame. It is easy to show
that when $v_s=0$, the resulting flat-trapped density is
\begin{equation}
  \label{eq:nft}
  n_f=  {2\over \sqrt{\pi}}|\phi|^{1/2}+
  \etothe{|\phi|}{\rm erfc}(|\phi|^{1/2}).
\end{equation}
Poisson's equation for the ion hole is then
\begin{equation}
  \label{eq:Poisson}
  {d^2\phi\over dz^2} = n_e-n_f-\tilde n_i
  = \etothe{\phi/\theta}- [{2\over \sqrt{\pi}}|\phi|^{1/2}+
  \etothe{|\phi|}{\rm erfc}(|\phi|^{1/2})] -\tilde n_i,
\end{equation}
where $\tilde n_i=\int \tilde f_i dv$ is the hole trapped ion density
deficit.  Regard this equation as giving an expression for
$\tilde n_i$ that consists of three terms: the electron density
$n_e=\etothe{\phi/\theta}$, minus the flat-trapped ion density
$n_f= [{2\over \sqrt{\pi}}|\phi|^{1/2}+ \etothe{|\phi|}{\rm
  erfc}(|\phi|^{1/2})]$, plus the field divergence, which we denote
$n_\phi\equiv-{d^2\phi\over dz^2}$. That is, $\tilde n_i=
n_\phi+n_e-n_f$.


If the hole potential $\phi(z)$ is prescribed, then the divergence
term is known, and $\tilde n_i(z)$ is given by Poisson's equation. The
ion distribution deficit $\tilde f_i$, which is a function only of energy,
must be such as to give rise to this $\tilde n_i$. The (Abel) integral
equation governing $\tilde f_i$ may be inverted to obtain
\begin{equation}
  \label{eq:abel}
  \tilde f_i(v_\psi) = {1\over \sqrt{2}\pi}\int_0^{|\psi|-v_\psi^2/2}
    {d\tilde n_i\over d|\phi|}{d|\phi|\over
      \sqrt{|\psi|-v_\psi^2/2-|\phi|} }
    ={1\over \sqrt{2}\pi}\int_0^{-\energy}
    {d\tilde n_i\over d|\phi|}{d|\phi|\over
      \sqrt{-\energy-|\phi|} },
\end{equation}
where $v_\psi$ denotes the ion velocity at the center of the hole
where $\phi=\psi$, that is $v_\psi=\sqrt{2(\energy+|\psi|)}$. Since
eq.\ (\ref{eq:abel}) is linear in the density deficit, we can treat the
three density terms separately and then add them up. The electron
derivative is ${dn_e\over d|\phi|}= -\etothe{-|\phi|/\theta}/\theta$,
and the flat-trapped contribution is
${dn_f\over d|\phi|}= \etothe{|\phi|}{\rm erfc}(|\phi|^{1/2})$.

As an illustrative choice let us suppose that the hole potential is of
the form
\begin{equation}
  \label{eq:potl}
  \phi(z) = \psi \sech^\ell\left({z\over\ell\lambda}\right).
\end{equation}
The parameter $\ell$ is usually taken to be 4, but we retain
slightly more generality, because we can still derive closed analytic
expressions. That mathematical convenience arises from the identity
${d^2\over dx^2} \sech^\ell x = \ell\sech^\ell x
-\ell(\ell+1)\sech^{\ell+2}x$, from which we find (for negative
$\phi$, $\psi$)
\begin{equation}
  \label{eq:diverg}
  -n_\phi={d^2\phi\over dz^2}=-{1\over \lambda^2}\left[
    |\phi| -{\ell+1\over \ell}{|\phi|^{(\ell+2)/\ell}\over |\psi|^{2/\ell}}
  \right],
\end{equation}
and so
\begin{equation}
  \label{eq:derivdiv}
  {d n_\phi\over d|\phi|}={1\over \lambda^2}\left[
    1 -{\ell+1\over \ell}{\ell+2\over \ell}\left|\phi\over \psi\right|^{2/\ell}
  \right].
\end{equation}

To avoid divergent slope of $\tilde f_i$ at $\energy=0$, it is necessary
that $\left.d\tilde n_i\over d\phi\right|_{\phi=0}=0$. That condition 
immediately becomes
\begin{equation}
  \label{eq:shielding}
 0= \left[{d n_\phi\over d|\phi|} +  {dn_e\over d|\phi|}- {dn_f\over d|\phi|}\right]_{\phi=0}=
  {1\over \lambda^2}- {1\over \theta} - 1.
\end{equation}
Thus the parameter $\lambda$ must be given by
$ {1\over \lambda^2}= {1\over \theta}+1$, and it becomes precisely the
screening length combining electron and ion Debye lengths in ion Debye
length units, because $\lambda_{De}^2=\lambda_{Di}^2\theta$, and we
are working in $\lambda_{Di}$ units.

Substituting the three contributions $dn_e\over d|\phi|$,
$dn_f\over d|\phi|$, and $|dn_\phi\over d|\phi|$ into
eq.\ (\ref{eq:abel}) we can find three corresponding contributions to
$\tilde f_i=\tilde f_{ie}+\tilde f_f+\tilde f_\phi$. After
considerable integration effort they can be expressed in terms of the
complex Faddeeva function\footnote{The Faddeeva function is related to
  the Plasma Dispersion function via $Z(z)=i\sqrt{\pi} w(z)$ and
  to the Dawson integral function $F(z)$ by $w(iz)=?2F(z)$.}
$w(z)\equiv\etothe{-z^2}{\rm erfc}(-iz)$ , and the
Gamma function $\Gamma(z)$, as follows:
\begin{equation}
  \label{eq:fie}
  \tilde f_{ie}=\pm{1\over \sqrt{2\pi}}\Im[\theta^{-1/2}w(\sqrt{|\energy|/\theta})],
\end{equation}
\begin{equation}
  \label{eq:ff}
  \tilde f_{f}={1\over \sqrt{2\pi}}\Re[1-w(i\sqrt{|\energy|})],
\end{equation}
\begin{equation}
  \label{eq:fphi}
  \tilde f_\phi=-{\sqrt{|\energy|}\over \sqrt{2\pi}\lambda^2}
  \left[{2\over \sqrt{\pi}} -
    {\ell+1\over \ell}{\ell+2\over \ell}
    {\Gamma(1+2/\ell)\over \Gamma(3/2+2/\ell)}
    \left|\energy\over \psi\right|^{2/\ell}
    \right].
\end{equation}
Eqs. (\ref{eq:fie}) and (\ref{eq:ff}) are equivalent to terms found by
Chen et al. But they used a Gaussian potential shape so their
$\tilde f_\phi$ is different, and they used electron Debye length
units. By virtue of the condition (\ref{eq:shielding}), the
coefficient of $\sqrt{\energy}$ in $\tilde f_i$ cancels to zero
between the three terms. However, the term
$\propto |\energy|^{1/2+2/\ell}$ (from $\tilde f_\phi$) remains. If
$\ell>4$ its derivative diverges as $|\energy|\to 0$, which is mildly
unphysical. But it avoids the more implausible non-monotonic behavior
that a Gaussian potential form has. The standard value $\ell=4$ makes
the bracket $[2/\sqrt{\pi} -(15\sqrt{\pi}/16)|\energy/\psi|^{1/2}]$,
and $\tilde f'(0)$ is finite. For $l<4$, $f'(0)=0$. These results make
no approximations, but assume that $f(v)$ is a function only of
$\energy$, i.e.\ satisfies the steady Vlasov equation. They are
illustrated later in Fig.\ \ref{thetaminplot}.


\section{Shifted Maxwellian Ions}

If the ion Maxwellian is shifted by $v_s\not=0$ in the frame of the
hole, then there exists no useful closed form expression for
$\tilde n_f$ at arbitrary $\phi$. Numerical curves of $\tilde n_f$
have been given elsewhere[]. They continuously decrease as $v_s$
increases. Their slope at $\phi=0$ can be shown by expansion to be
${dn_f\over d|\phi|}=-{1 \over 2} Z'(v_s/\sqrt{2})$, which is what
determines the attracted species distant shielding length.  Therefore eq.\
(\ref{eq:shielding}) is modified to become
\begin{equation}
  \label{eq:modshield}
0={1\over \lambda^2}- {1\over \theta} + {1 \over 2} Z'(v_s/\sqrt{2}).  
\end{equation}

This condition (or its equivalent) at $\phi\to0$ has been referred to,
in many publications based on prescribing $\tilde f(\energy)$ rather
than $\phi(z)$, as the ``nonlinear dispersion relation''. It is then
supposed that it relates the hole's potential amplitude, speed, and
trapped particle temperature. That is a misleading perspective,
whether for electron holes or ion holes, because it adopts a
particular form for the trapped distribution (negative temperature
Maxwellian), which strongly constrains the potential shape. Actually
the relative shape of the hole can have a whole range of widths upward
from a minimum, provided $\phi$ decays at large distances (small
$\phi$) with second derivative length scale ($\lambda$) satisfying
eq.\ (\ref{eq:modshield}), and the trapped deficit $\tilde f$ is
consistent with the rest of the potential shape. A hole's speed must
not be so great that the combined particle terms
$-{dn_{reflected}\over d|\phi|}+{dn_{flattrap}\over d|\phi|}$ becomes
negative, otherwise (\ref{eq:modshield}) cannot be satisfied. That
requires $-Z'(v_s/\sqrt{2})/2>-1/\theta$. Speed must also be small
enough that the background attracted species distribution is non-zero
otherwise there is no phase-space density there, in which it could be
a hole. The hole amplitude $\psi$ is limited only by the
non-negativity of $f(v)$, which has only weak dependence on speed and
width at typical potential amplitudes.

A related misunderstanding is the belief, often repeated, that for an
ion hole to exist (``a solution of the nonlinear dispersion relation
[to] exist[s]'') the temperature ratio must satisfy
$T_e/T_i=\theta > 1/|{\rm min}(-Z'(v_s/\sqrt{2})/2)|\simeq 3.5$. The
minimum of $-Z'/2$ over all real arguments is -0.285 which is where
the value 3.5 comes from. But in fact there is a sign error in this
published criterion and at the $-Z'/2$ minimum eq.\
(\ref{eq:modshield}) requires $-0.285>-1/\theta$, which is
$\theta < 3.5$. In any case, the minimum of $-Z'/2$ occurs at a speed
%1.506*sqrt(2)
$v_s= 2.13$; so the criterion applies only for the worst case of a
high speed hole, and not at all to more typical holes lying deeper
within the attracted species' bulk population, where $-Z'/2$ is
positive. The relevant effect of finite response of the repelled
species is instead that it increases the required trapped species
deficit $|\tilde f_i|$, and makes the requirement of non-negativity
more stringent. Comprehensive calculations that are noted to provide
counter-examples to the erroneous criterion are given by Chen et al[].

Detailed calculations of hole structure with $\sech^l$ potential shape
and unshifted distributions, illustrating the actual hole feasibility,
are given in Fig.\ \ref{thetaminplot}.
\begin{figure}[htp]
  \includegraphics[width=.5\hsize]{ftofvel}\hskip-1.5em(a)
  \includegraphics[width=.47\hsize]{thetamin}\hskip-1.5em(b)
\caption{Examples of ion (or electron) holes with unshifted background
  Maxwellian distributions of both species. (a) the required trapped
  deficit $\tilde f(v)$ to produce a hole of the form
  $|\phi|=\psi\sech^l(z/l\lambda)$ with $l=$ 2, 4, or 8, for two ratios of
  the attracted to repelled species ratio ($T_a/T_e$). (b) The
  particle $\tilde f_{part}=\tilde f_f+\tilde f_{ie}$ and potential
  $\tilde f_\phi$ contributions, in combinations $\sqrt{2\pi}\tilde
  f_{part}$ and $-1-\sqrt{2\pi}\tilde f_\phi$ which are equal
  at the maximum allowable value of $T_a/T_r$ given the
  various amplitudes $\psi$.\label{thetaminplot}}
\end{figure}
The mathematics (equations \ref{eq:fie} to \ref{eq:fphi}) is identical
for ion and electron holes, when expressed in units normalized to the
ion or electron parameters respectively and $T_a/T_r$ the ratio of
attracted to repelled species temperature. In these normalized units,
non-negativity of the trapped distribution requires that
$\sqrt{2\pi}\tilde f > -1$, marked with a dotted line in Fig.\
\ref{thetaminplot}(a). Thus, of the six cases shown there, only the
$l=2$ case with equal temperatures $T_a/T_r=1$ is impossible. All the
others are possible equilibria. If $\psi$ were decreased, none would
be impossible; if it were increased more cases would become
unphysical. Fig.\ \ref{thetaminplot}(b) illustrates separately the
particle (solid line) and potential (dashed line) terms that
contribute to determining the non-negativity criterion. Where the
lines for the same $\psi$ value intersect determines the
maximum permissible value of $T_a/T_r$ which for an ion hole is
$T_i/T_e=1/\theta$. Clearly, a wide range of this ratio from $10^{-1}$
to $10^{+1}$ and beyond is permissible depending on the value of
$\psi$.

If the attracted species has asymmetry in the form of a velocity shift
$v_s$ relative to the hole, the main effect is a reduction in its
density response to the hole potential ${dn_f\over d|\phi|}$ and the
contribution $\tilde f_f$ to the required trapped depletion. But it
also reduces the level of the flat trapped region to
$f_f={1\over \sqrt{2\pi}}\exp(-v_s^2/2)$. As $v_s$ is increased,
therefore, the hole amplitude must be reduced (eventually to $\sim 0$)
to avoid trapped deficit corresponding to negative distribution
function. 

It has been shown elsewhere [] that electron holes with unshifted
Maxwellian distributions of ions and electrons are actually unstable
to self-acceleration. Ion holes of the same type are also unstable to
such self-acceleration, but the growth rate is far smaller. So they
remain in near-equilibrium much longer. For much the same reasons,
holes in a shifted Maxwellian repelled species background distribution
are in general not steady equilibria, because they experience a net
force exerted by the reflected repelled species. Instead they
accelerate[]. The changing hole velocity (relative to attracted
background) causes evolution also in the hole amplitude, and as the
hole moves away from the center of the attracted species' distribution
peak it decays in depth until eventually it collapses.



\bibliography{JabRef}


\end{document}


%%% Local Variables:
%%% mode: latex
%%% TeX-master: t
%%% End:
